\documentclass[a4paper,12pt]{refrep}
\title{aa-kernel}
\usepackage{color}
\definecolor{light-gray}{gray}{0.95}
\newcommand{\code}[1]{\colorbox{light-gray}{\texttt{#1}}}

\begin{document}
\chapter{\code{display\_surface} - An HDMI Framebuffer Peripheral}

\section{Abstract}
\code{display\_surface} is a video display peripheral comprising an 
HDMI transmitter and a custom AXI master, 
automatically accessing a framebuffer located in host memory 
and displaying it over a TMDS channel. This project was developed
using the Digilent Zybo, and is capable of rendering a frame 
or sequence of frames with next to no load on the CPU.

\section{Architecture}
\subsection{Clocking}
The clocks for the AXI master and the TMDS encoder may differ. 
This clock crossing is resolved by alternating banks 
of a dual-port block RAM functioning as a FIFO.

The timing of the switch-over of these banks is constrained 
such that the stream of output pixels to the transmitter never 
underruns.

\subsection{The TMDS Encoder}
The TMDS encoder (see \code{tmds\_channel.v}) is a 4-stage pipelined 
module implementing the algorithm outlined in the Digital 
Video Interface specification. 

\subsection{The AXI Master - Framebuffer DMA}
The AXI Master feeds the HDMI transmitter FIFO with data from host 
memory (DRAM). It issues AXI burst transactions whenever space is
available in the FIFO, and when the end of the screen is reached,
it either starts on the next framebuffer or loops back to the 
current framebuffer. It may also send an interrupt to the CPU, 
if enabled.
\subsection{The AXI Slave - Configuration Registers}
The AXI config registers are the means by which the CPU provides a 
4k-aligned address to read the framebuffer(s) from, as well as select
double-buffering and enable interrupts. 


\end{document}
